% http://www.cse.iitk.ac.in/users/braman/students/good-report.html
\documentclass[english,a4paper]{scrartcl}

\usepackage[utf8]{inputenc}
\usepackage[english]{babel}
\usepackage{graphicx}
\usepackage{xcolor}
\usepackage{fullpage}
\usepackage{parskip}
\usepackage{newclude}
\usepackage{hyperref}
\usepackage{listings}

\hypersetup{
    %bookmarks=true,         % show bookmarks bar?
    unicode=false,          % non-Latin characters in Acrobat's bookmarks
    pdftoolbar=true,        % show Acrobat's toolbar?
    pdfmenubar=true,        % show Acrobat's menu?
    pdffitwindow=false,     % window fit to page when opened
    pdfstartview={FitH},    % fits the width of the page to the window
    pdftitle={TDT4171 Artificial Intelligence Methods - Assignment 4},
    % title
    pdfauthor={Stian Hvatum},     % author
    pdfsubject={TDT4171 Artificial Intelligence Methods},   % subject of the document
    pdfcreator={Stian Hvatum},   % creator of the document
    pdfproducer={Stian Hvatum}, % producer of the document
    pdfnewwindow=true,      % links in new window
    colorlinks,       % false: boxed links; true: colored links
    linkcolor=black,          % color of internal links
    citecolor=green,        % color of links to bibliography
    filecolor=magenta,      % color of file links
    urlcolor=cyan           % color of external links
}
\title{TDT4171 Artificial Intelligence Methods\\
\Small Exercise 4\\
\Huge Decision Tree Learning Algorithms}
\author{Stian Hvatum (hvatum)\\MTDT}

\lstset{
    keywordstyle=\bfseries\ttfamily\color[rgb]{0,0,1},
    identifierstyle=\ttfamily,
    commentstyle=\color[rgb]{0.133,0.545,0.133},
    stringstyle=\ttfamily\color[rgb]{0.627,0.126,0.941},
    showstringspaces=false,
    basicstyle=\tiny,
    numberstyle=\tiny,
    framexleftmargin=3pt,
    numbers=none,
    stepnumber=1,
    numbersep=15pt,
    tabsize=4,
    breaklines=true,
    prebreak = \raisebox{0ex}[0ex][0ex]{\ensuremath{\hookleftarrow}},
    breakatwhitespace=false,
    aboveskip={1.5\baselineskip},
    columns=fixed,
    upquote=true,
    extendedchars=true,
  	frame=tb,
    sensitive=true,
}

\begin{document}
\maketitle
\thispagestyle{empty}
\newpage
\pagenumbering{Roman}
\tableofcontents
\newpage
\pagenumbering{arabic}
\section{Intro}
In this assignment I have implemented the Decision Tree Learner Algorithm from
AIMA Figure 18.5. I have chosen to implement the algorithm in C, because we are
learning C in different cources at school right now, and C is a good and
efficient language to implement this kind of algorithms, with good support for
math and data structures.

\section{How to build the program}
The program is provided with builds for Windows, Linux and OSX. Binaries are
located in the {\ttfamily Releases/\emph{system}}-folder, and named
{\ttfamily tree-learning}, where system is the provided platforms. To compile
the source files for a different platform than provided, you can run {\ttfamily
make clean \&\& make}. You will then have a fresh binary file in the {\ttfamily
bin}-directory. I have tested the code on {\ttfamily login.stud.ntnu.no},
{\ttfamily asti.idi.ntnu.no}, a Windows 7 machine with MSYS and MinGW, and with
current version of X-Code on Max OSX Lion. All with GCC, but the code will
probably compile with Visual Studio as well.

\section{How to run the program}
The program is run from the command line. You can supply input files, test data,
which algorithm (Random or Information Gain) and iterations. How to supply each
of the arguments are given by the program help file:\\

\begin{ttfamily}
\begin{tabbing}
Help for \=Decision Tree Learner\\
\>bin/tree\=-learning.exe [A\=RGUMENTS] {INPUT FILES}\\
\\
Arguments:	\>\>Short form:	\>Desc:\\
--iterations n	\>\>-i n		\>n is the number of times the input files are read.\\
--random	\>\>-r		\>If set, the random-gain is used instead\\\>\>\> of the
information gain algorithm\\
--test file	\>\>-t file		\>file is the a file to check our tree against\\

--help	\>\>-h		\>Displays this help text and exits\\
\\
Example:\\
\>bin/tree-learning.exe -r -i 100 -t test.txt training.txt
training2.txt\\ 
\end{tabbing}
\end{ttfamily}
 
\newpage
\section{Results}
\subsection{Resulting tree from Information Gain}
This is the resulting tree gained from the training set. Each \emph{Val} is an
edge, each node is either a \emph{Class} or an \emph{Attribute}, where 
\emph{Class}es are leaf nodes. This tree is correct on 26 of the 28 examples
in the test-set. The tree matches all examples in its own training set.
\begin{lstlisting}
Attrib:1
 | Val:1 Class:1
 | Val:2 Attrib:5
 |  | Val:1 Attrib:3
 |  |  | Val:1 Attrib:2
 |  |  |  | Val:1 Class:1
 |  |  |  | Val:2 Class:2
 |  |  | Val:2 Attrib:2
 |  |  |  | Val:1 Class:2
 |  |  |  | Val:2 Class:1
 |  | Val:2 Attrib:6
 |  |  | Val:1 Attrib:2
 |  |  |  | Val:1 Attrib:3
 |  |  |  |  | Val:1 Class:1
 |  |  |  |  | Val:2 Class:2
 |  |  |  | Val:2 Attrib:3
 |  |  |  |  | Val:1 Class:2
 |  |  |  |  | Val:2 Class:1
 |  |  | Val:2 Attrib:4
 |  |  |  | Val:1 Attrib:2
 |  |  |  |  | Val:1 Attrib:3
 |  |  |  |  |  | Val:1 Class:1
 |  |  |  |  |  | Val:2 Class:2
 |  |  |  |  | Val:2 Attrib:3
 |  |  |  |  |  | Val:1 Class:2
 |  |  |  |  |  | Val:2 Class:1
 |  |  |  | Val:2 Attrib:2
 |  |  |  |  | Val:1 Class:2
 |  |  |  |  | Val:2 Attrib:3
 |  |  |  |  |  | Val:1 Class:2
 |  |  |  |  |  | Val:2 Class:1

\end{lstlisting}
\newpage
\subsection{Resulting tree from Random}
Here we see one possible tree from the Random algorithm. There Random trees are
usually very long ones, so I ran the program a few time to get one that is short
enough to fit on one page. This tree is correct on 21 of the 28 examples
in the test-set, but this is about arbitary for each random tree. Though every
Random tree is 100\% match on its own training set, just as the Information Gain
trees.
\begin{lstlisting}
Attrib:1
 | Val:1 Class:1
 | Val:2 Attrib:4
 |  | Val:1 Attrib:5
 |  |  | Val:1 Attrib:2
 |  |  |  | Val:1 Attrib:3
 |  |  |  |  | Val:1 Class:1
 |  |  |  |  | Val:2 Class:2
 |  |  |  | Val:2 Attrib:3
 |  |  |  |  | Val:1 Class:2
 |  |  |  |  | Val:2 Class:1
 |  |  | Val:2 Attrib:7
 |  |  |  | Val:1 Attrib:2
 |  |  |  |  | Val:1 Class:1
 |  |  |  |  | Val:2 Attrib:3
 |  |  |  |  |  | Val:1 Class:2
 |  |  |  |  |  | Val:2 Class:1
 |  |  |  | Val:2 Attrib:6
 |  |  |  |  | Val:1 Attrib:2
 |  |  |  |  |  | Val:1 Attrib:3
 |  |  |  |  |  |  | Val:1 Class:1
 |  |  |  |  |  |  | Val:2 Class:2
 |  |  |  |  |  | Val:2 Attrib:3
 |  |  |  |  |  |  | Val:1 Class:2
 |  |  |  |  |  |  | Val:2 Class:1
 |  |  |  |  | Val:2 Attrib:3
 |  |  |  |  |  | Val:1 Attrib:2
 |  |  |  |  |  |  | Val:1 Class:1
 |  |  |  |  |  |  | Val:2 Class:2
 |  |  |  |  |  | Val:2 Class:2
 |  | Val:2 Attrib:5
 |  |  | Val:1 Attrib:6
 |  |  |  | Val:1 Attrib:3
 |  |  |  |  | Val:1 Attrib:7
 |  |  |  |  |  | Val:1 Attrib:2
 |  |  |  |  |  |  | Val:1 Class:1
 |  |  |  |  |  |  | Val:2 Class:2
 |  |  |  |  |  | Val:2 Attrib:2
 |  |  |  |  |  |  | Val:1 Class:1
 |  |  |  |  |  |  | Val:2 Class:2
 |  |  |  |  | Val:2 Attrib:2
 |  |  |  |  |  | Val:1 Class:2
 |  |  |  |  |  | Val:2 Class:1
 |  |  |  | Val:2 Attrib:7
 |  |  |  |  | Val:1 Class:1
 |  |  |  |  | Val:2 Attrib:2
 |  |  |  |  |  | Val:1 Attrib:3
 |  |  |  |  |  |  | Val:1 Class:1
 |  |  |  |  |  |  | Val:2 Class:2
 |  |  |  |  |  | Val:2 Class:1
 |  |  | Val:2 Attrib:3
 |  |  |  | Val:1 Attrib:2
 |  |  |  |  | Val:1 Class:1
 |  |  |  |  | Val:2 Class:2
 |  |  |  | Val:2 Attrib:6
 |  |  |  |  | Val:1 Attrib:2
 |  |  |  |  |  | Val:1 Class:2
 |  |  |  |  |  | Val:2 Class:1
 |  |  |  |  | Val:2 Attrib:7
 |  |  |  |  |  | Val:1 Attrib:2
 |  |  |  |  |  |  | Val:1 Class:2
 |  |  |  |  |  |  | Val:2 Class:1
 |  |  |  |  |  | Val:2 Class:2


\end{lstlisting}

\subsection{Random vs. Information Gain}
The Random trees also have ``random'' accuracy, but they are generally less
accurate than those created by the Information Gain algorithm. Occationally the
Random-created ones yields a better accuracy, but this is due to the fact that
Information Gain uses a heuristic on entropy to calculate which attributes are
the most important ones, and a random selection of attributes might a seldom
time perform better at finding the right ones.

I will state that the Information Gain algorithm is the better one, as it is
usually generates better learning trees than the Random algorithm.

\subsection{Multiple runs with Random}
If I run Random multiple times with same input, I will get a different result as
good as every time. The Random algorithm is seeded with current time as the
program starts, and will choose a different {\ttfamily
most-important}-attributes on each run where seconds on the host computer clock
has changed\footnote{This is implementation dependent. A real Random-gain would
be different on each run} since last run.

\subsection{Multiple runs with Information Gain}
If I multiply the input for the Information Gain algorithm, I still end up with
the exact same output. This is due to the fact that there are no more unique
examples, and the new equal ones are eaten in chunks by the second check in the
algorithm: {\ttfamily \textbf{if} \textit{all examples} have same class 
\textbf{then return} the class}. The examples are in that way treated as a
mathematical set rather than a collection. If we increase the training set with
new unique examples rather than reusing the old ones, we would gain better
results. Eg. if we concatenate the training set and the test set, we will get
another learning tree which is 100\% accurate on both those sets.

\section{Small Code Guide}
The program is split in several {\ttfamily .c}-files, each representing a part
of the program. Most of the math stuff is done in {\ttfamily importance.c}. Here
you will find the two implementations, Information Gain and Random. The
\textbf{B}-function is also found here, named {\ttfamily func\_b}. There is also
som algorithm-dependent math in {\ttfamily plurality\_value.c}, where we simply
find which class is most representeted in a set of examples.

The main algorithm for building the tree is found in {\ttfamily
decision\_tree\_learning.c}, and contains a lot of help-functions to build the
tree. The main function is {\ttfamily decision\_tree\_learning}, which except
for some C-specific malloc's and free's  are a pure copy of Fig 18.5 from AIMA.

Code found in {\ttfamily types.c} is mainly data structures to build the
examples and the trees, while the code in {\ttfamily main.c} is the reading of
input files, initializing stacks and tree-pointers and cleanup upon completion.
\end{document}
