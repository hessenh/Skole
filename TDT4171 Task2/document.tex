\documentclass[english]{article}

% For norske bokstaver
% \usepackage[T1]{fontenc}
\usepackage{babel}
\usepackage[utf8]{inputenc}
\usepackage{parskip} % for litt mellomrom mellom avsnittene
\usepackage{amsmath}
\usepackage{tocloft}
\usepackage{graphicx}
\usepackage{subfig}

\renewcommand{\thesection}{Part \Alph{section}}
\renewcommand{\thesubsection}{\arabic{subsection}}

\title{TDT4171 Artificial Intelligence Methods\\
\Huge Exercise 2}
\author{Stian Hvatum (hvatum)\\MTDT}

\begin{document}
\maketitle
\pagenumbering{arabic}
\newpage
\section{The Umbrella domain as an HMM}
\begin{itemize}
  \item The set of unobserved variable for a given time-slice \(t\) is the single variable \(Rain_t\).
  \item The set of observed variables, or evidence, for a given time-slice is the single variable \(Umbrella_t\).
  \item The \emph{dynamic model} P\((X_t|X_{t-1})\) is:
\[ \left(
   \begin{array}{cc}
    0.7 & 0.3 \\
	0.3 & 0.7
	\end{array}
\right)\] 
where each \(value_{i,j}\) is the probability for changing from state \(i\) to
state \(j\).
  \item The \emph{observation model} P\((E_t|X_t)\) is:
\[ \left(
   \begin{array}{cc}
    0.9 & 0 \\
	0 & 0.2
	\end{array}
\right)\]
where the observation \(Umbrella_t\) is \emph{true}, and 
\[ \left(
   \begin{array}{cc}
    0.1 & 0 \\
	0 & 0.8
	\end{array}
\right)\] 
where the observation is \emph{false}. Notice that the matrix for \emph{false}
is the \emph{identity}-matrix - the \emph{true}-matrix.

\item In this model, we assume that the process is a \emph{stationary process}.
A \emph{stationary process} is a process where the laws that transist the state
from one to another never changes. In this specific case, this means that we
assume that the probaility for rain today given rain (or not rain) yesterday is
always the same. Yet we also assumes that rain today only depends on rain
yesterday, thus making it a \emph{first-order Markov process}.

These assumptions are probably not sufficient for a weather-forecasting-system,
but introducing either more sensors or more orders to the Markov process will
add much complexity and add little value to this example scenario.
\end{itemize}

\newpage
\section{Filtering with \texttt{FORWARD}}
\subsection{}
The calculation is given in the MATLAB-file \emph{partB1.m}. Since MATLAB
indexing starts with 1 and not 0, the probabilities for day 2 is in
\(ans(1,1,3)\). Answer is 0.88336.

\subsection{}
These calculations are given in the MATLAB-file \emph{partB2.m}.
\begin{figure}[h]
\begin{tabular}{|l||c|c|}
\hline
Day & \(Rain\) & \(\lnot Rain\) \\
\hline
\hline
0 & 0.50000 & 0.50000\\
1 & 0.81818 & 0.18182\\
2 & 0.88336 & 0.11664\\
3 & 0.19067 & 0.80933\\
4 & 0.73079 & 0.26921\\
5 & 0.86734 & 0.13266\\
\hline
\end{tabular}
\caption{Normalized forward-messages}
\end{figure}

\newpage
\section{Smoothing using \texttt{FORWARD-BACKWARD}}
\subsection{}
The calculation is given in the MATLAB-file \emph{partC1.m}. Since MATLAB
indexing starts with 1 and not 0, the probabilities for day 2 is in
\(ans(:,:,3)\), and is \(\langle 0.88336, 0.11664\rangle\).

\subsection{}
These calculations are given in the MATLAB-file \emph{partC2.m}.
\begin{figure}[h]
\begin{tabular}{|l||c|c|}
\hline
Day & \(Rain\) & \(\lnot Rain\) \\
\hline
\hline
0 & 0.64694 & 0.35306\\
1 & 0.59232 & 0.40768\\
2 & 0.37627 & 0.62373\\
3 & 0.65334 & 0.34666\\
4 & 0.62727 & 0.37273\\
5 & 0.50000 & 0.50000\\
\hline
\end{tabular}
\caption{Normalized Backward-messages from \texttt{FORWARD-BACKWARD}}
\end{figure}
\begin{figure}[h]
\begin{tabular}{|l||c|c|}
\hline
Day & \(Rain\) & \(\lnot Rain\) \\
\hline
\hline
0 & 0.044385 & 0.024223\\
1 & 0.066118 & 0.045508\\
2 & 0.090639 & 0.150251\\
3 & 0.45930 & 0.24370\\
4 & 0.69000 & 0.41000\\
5 & 1.00000 & 1.00000\\
\hline
\end{tabular}
\caption{Backward-messages from \texttt{FORWARD-BACKWARD}}
\end{figure}

\newpage
\section{Code}
All code is written for Octave, but should also work with MATLAB
\footnote{All code runs successfully on MATLAB R2011b x64. Answers
given in this report is calculated with Octave 3.2.4.}.
The files included should be:
\begin{itemize}
  \item forward.m
  \item backward.m
  \item normalize.m
  \item forwardbackward.m
  \item partB1.m
  \item partB2.m
  \item partC1.m
  \item partC2.m
\end{itemize}

\end{document}
