\documentclass[norsk,a4paper,landscape]{article}

% For norske bokstaver
\usepackage[T1]{fontenc}
\usepackage{babel}
\usepackage{wrapfig}
\usepackage{listings}
\usepackage{color}
\usepackage[a4paper,landscape]{geometry}
\usepackage[utf8]{inputenc}
\usepackage{parskip} % for litt mellomrom mellom avsnittene
\usepackage{amsmath,bussproofs}

\title{TDT4136 Logic And Reasoning Systems\\
{\Huge Exercise 3}}
\author{Stian Hvatum (hvatum) og Bjørn Christian Seime (bjorncs)\\MTDT}

\begin{document}
\maketitle
\thispagestyle{empty}
\newpage
\pagenumbering{Roman}
\tableofcontents
\newpage
\pagenumbering{arabic}
\section{Task 1}
\framebox[\linewidth][c]{\parbox{0.9\linewidth} {
A steeple-building robot looks for two blocks that are clear and puts one of
them on top of the other (if the block being moved can be located as being on some other block or on the floor).
\newline
Write down the FOL expression that might be used to determine whether or not
there exists two such blocks. } }

 
\(\exists x,y Box(x) \land Box(y) \land (On\_floor(y) \lor On\_top(y)) \land
(On\_floor(x) \lor On\_top(x) \land \lnot (x = y))
\)

\section{Task 2}
\framebox[\linewidth][c]{\parbox{0.9\linewidth} {
The function \(Cons(x,y)\) denotes the list formed by inserting the element
\(x\) at the head of the list \(y\). We denote the empty list [] by \(Nil\);
the list [2] by \(Cons(2,Nil);\) the list [1,2] by \(Cons(1,Cons(2,Nil));\) and
so on. The formula \(Last(l,e)\) is intended to mean that \(e\) is the last
element of the list \(l\). We have the following axioms:

 \( \forall u (Last(Cons(u,Nil),u)) \)

 \( \forall x,y,z (Last(y,z) \Rightarrow Last(Cons(x,y),z)) \)
}  }

\subsection{Prove the following theorem from these axioms by the method of
resolution refutation}
\framebox[\linewidth][c]{\parbox{0.9\linewidth} {
\(\exists v Last(Cons(2,Cons(1,Nil)),v) \)
} }

Vi har aksiomene: \\
\( \forall u (Last(Cons(u,Nil),u)) \) \\
 \( \forall x,y,z (Last(y,z) \Rightarrow Last(Cons(x,y),z)) \) \\

Vi begynner med å neglere påstanden: \\
\(\lnot(\exists v Last(Cons(2,Cons(1,Nil)),v)) \) \\

Vi har nå disse uttrykkene: \\
\(  \forall u (Last(Cons(u,Nil),u)) \\
    \forall x,y,z (Last(y,z) \Rightarrow Last(Cons(x,y),z)) \\
    \lnot(\exists v Last(Cons(2,Cons(1,Nil)),v)) \) \\

Vi erstatter kvantifikatorene med Skolem-funksjoner: \\
\( Last(Cons(D(u),Nil),D(u)) \\
     Last(B(y),C(z)) \Rightarrow Last(Cons(A(x),B(y)),C(z)) \\
    \lnot Last(Cons(2,Cons(1,Nil)),E(v)) \) \\

Så fjerner vi implikasjoner, og får KB: \\
\(Last(Cons(D(u),Nil),D(u))\\
  \lnot Last(B(y),C(z)) \lor Last(Cons(A(x),B(y)),C(z)) \) \\
      
og startutrykk: \\
\(\lnot Last(Cons(2,Cons(1,Nil)),E(v)) \) \\

Deretter viser vi at dette er en motsigelse: \\
\begin{tabular}{|c|c|c|}
\hline
\(
\lnot Last(B(y),C(z)) \lor Last(Cons(A(x),B(y)),C(z)) \) & \(\lnot
 Last(Cons(2,Cons(1,Nil)),E(v)) \) & \(\Theta
 \{A(x)/2,B(y)/Cons(1,Nil),C(z)/E(v)\}\)\\
\hline
\(Last(Cons(D(u),Nil),D(u))\) & \(\lnot Last(Cons(1,Nil),E(v)) \)  & \(\Theta
D(u)/E(v),E(v)/1\}\) \\
\hline 
 \ & [ ] & \ &
 \hline

\end{tabular}

\subsection{Use answer literal extraction to find v, the last element of the
\(list [2,1]\)}
Vi ser av siste setning i beviset at \(E(v)\) må være 1

\newpage
\section{Task 3}
\subsection{Create a Full-Adder}
\(XOR(i1, i2, oa) \land AND(i1, i2, ob) \land AND(ci, oa, oc) \land OR(ob, oc,
co) \land XOR(oa, ci, o) \Rightarrow ADD(i1, i2, ci, o, co)
\)

\subsection{Queries for verification}
Vi ville brukt alle mulige kombinasjoner av denne typen, og sjekket at de var
sanne: \\
 \(ADD(0,0,0,0,0) \) \\
\(ADD(1,1,1,1,1) \) \\
\(ADD(1,1,0,0,1) \) \\
\(ADD(1,0,0,1,0) \) \\

Og tilsvarende sjekket at utsagn av denne typen var falske: \\
 \(ADD(0,1,0,0,0) \) \\
\(ADD(1,1,1,0,1) \) \\
\(ADD(1,1,0,1,1) \) \\
\(ADD(1,0,0,0,0) \) \\

\section{Task 4}
\subsection{}
Bruk KB:
\lstinputlisting[language=prolog]{ex3t4-1.pl}
og kjør:\\
?- last(cons(2, cons(1, nil)), V).

Denne spørringen svarer V = 1

\subsection{}
Bruk KB:
\lstinputlisting[language=prolog]{ex3t4-2.pl}
Kjør: \\
?- cons(1,[],A), cons(2, A, B), last(B, V).

Denne spørringen svarer \\
A = [1] \\
B = [2, 1] \\
V = 1 \\
    
\subsection{}
Vi definerer programmet slik:
\lstinputlisting[language=prolog]{ex3t4-3.pl}

\subsection{}
Vi definerer programmet slik:
\lstinputlisting[language=prolog]{ex3t4-4.pl}
\end{document}