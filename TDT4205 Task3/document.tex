%%This is a very basic article template.
%%There is just one section and two subsections.
\documentclass{article}

\usepackage{graphicx}
\usepackage{listings}
\usepackage{color}
\usepackage{makeidx}
\usepackage{hyperref}
\usepackage{parskip}
\usepackage{multirow}

\makeindex

\hypersetup{
    %bookmarks=true,         % show bookmarks bar?
    unicode=false,          % non-Latin characters in Acrobat’s bookmarks
    pdftoolbar=true,        % show Acrobat’s toolbar?
    pdfmenubar=true,        % show Acrobat’s menu?
    pdffitwindow=false,     % window fit to page when opened
    pdfstartview={FitH},    % fits the width of the page to the window
    pdftitle={TDT4205 Compilers - Exercise 3 - hvatum},    % title
    pdfauthor={Stian Hvatum},     % author
    pdfsubject={TDT4205 Compilers},   % subject of the document
    pdfcreator={Stian Hvatum},   % creator of the document
    pdfproducer={Stian Hvatum}, % producer of the document
    pdfnewwindow=true,      % links in new window
    colorlinks,       % false: boxed links; true: colored links
    linkcolor=red,          % color of internal links
    citecolor=green,        % color of links to bibliography
    filecolor=magenta,      % color of file links
    urlcolor=cyan           % color of external links
}

\lstset{ %
  literate= {+}{{$+$}}1 {*}{{$*$}}1 {=}{{$\gets$}}1 
            {<=}{{$\leq$}}1 {>=}{{$\geq$}}1 {!=}{{$\neq$}}1 
            {==}{{$\equiv$}}1 {=>}{{$\leadsto$}}1
            {->}{{$\rightarrow$}}1
}

\definecolor{listinggray}{gray}{0.9}
\definecolor{lbcolor}{rgb}{0.9,0.9,0.9}
\lstset{
    keywordstyle=\bfseries\ttfamily\color[rgb]{0,0,1},
    identifierstyle=\ttfamily,
    commentstyle=\color[rgb]{0.133,0.545,0.133},
    stringstyle=\ttfamily\color[rgb]{0.627,0.126,0.941},
    showstringspaces=false,
    basicstyle=\tiny,
    numberstyle=\tiny,
    framexleftmargin=3pt,
    numbers=left,
    stepnumber=1,
    numbersep=15pt,
    tabsize=4,
    breaklines=true,
    prebreak = \raisebox{0ex}[0ex][0ex]{\ensuremath{\hookleftarrow}},
    breakatwhitespace=false,
    aboveskip={1.5\baselineskip},
    columns=fixed,
    upquote=true,
    extendedchars=true,
  	frame=l,
    sensitive=true,
}

\renewcommand{\thesection}{PART \arabic{section}}
\renewcommand{\thesubsection}{Task \arabic{section}.\arabic{subsection}}
\renewcommand{\thesubsubsection}{\arabic{section}.\arabic{subsection}.\arabic{subsubsection}}

\title{TDT4205 Compilers\\
\Huge Exercise 3}
\author{Stian Hvatum (hvatum)\\MTDT}

\begin{document}
\maketitle

\section{Theory}
\subsection{Parsing}
\subsubsection{LL(k)}
Even if LL(\textit{k}) in theory can be extended with an $\rightarrow infinitly$
number of lookaheads, this will not resolve the problems with left recursion.
LL(\textit{k}) parses in a naive way, and if there exists a left recursion, like
S $\rightarrow$ Ss $| \epsilon$, it will allways match the first S, and will spin
generating more S's. This means we will continue beond the number of s's in the
parsed text, and things will go bad.

Even if we could modify the LL(\textit{k})  to recognize patterns using the
lookaheads, and match for a spesific number of recursive calls, we would run
into practical problems. For each lookahead-symbol extra we need, the parse
table grows, and at the end it will become humongous. To create and use such a
table is both space and time consuming, and we will never have neither space
nor time to parse languages and grammars with arbitary use of left recursion,
as it requires arbitary much space and time. Also, LL(\textit{k}) needs the
\textit{k} to be defined, wich sets an upper bound for the number of lookaheads
before we begin parsing. We may set the \textit{k} to 10000000000, and hope
that the language users will never exeed this number of recursions, but it will
not be a \emph{valid} parser for that language, as it can only handle a subset of
the possible language constructs.
\subsubsection{Left-recursive grammars}
The left-recursive grammar:\\
\begin{tabular}{ccl}
F & $\rightarrow$ & f I v A w S x \\ 
A & $\rightarrow$ & P \\
P & $\rightarrow$ & P I $| \epsilon$\\ 
S & $\rightarrow$ & S s $|$ s\\
I & $\rightarrow$ & i\\
\end{tabular}


The equivalent non-left-recursive grammer:\\
\begin{tabular}{ccl}
F & $\rightarrow$ & f I v A w S x \\ 
A & $\rightarrow$ & P \\
P & $\rightarrow$ & I P $| \epsilon$\\ 
S & $\rightarrow$ & s S'\\
S' & $\rightarrow$ & s S' $| \epsilon$\\
I & $\rightarrow$ & i\\
\end{tabular}

\subsubsection{{\ttfamily FIRST} and {\ttfamily FOLLOW} \& LL(1) Parsing table}
\subsubsection*{{\ttfamily FIRST} and {\ttfamily FOLLOW}}
Computing {\ttfamily FIRST}
\begin{itemize}
  \item f is in FIRST(F), since f is the first symbol in production of F, and f
  is a terminal, and thereby FIRST(f) = f.
  \item i is in FIRST(I), since FIRST(I) = FIRST(i) = i
  \item i is in FIRST(P), since FIRST(P) = FIRST(I) = i 
  \item $\epsilon$ is in FIRST(P), since P has a production P $\rightarrow
  \epsilon$
  \item i, $\epsilon$ is in FIRST(A), since FIRST(A) = FIRST(P) = i, $\epsilon$
  \item s is in FIRST(S) since FIRST(S) = FIRST(s) = s 
  \item s is in FIRST(S') since FIRST(S') = FIRST(s) = s 
  \item $\epsilon$ is in FIRST(S'), since S' has a production S' $\rightarrow
  \epsilon$
\end{itemize}
Computing {\ttfamily FOLLOW}
\begin{itemize}
  \item We start by adding \$ to F, since F is our \emph{start-symbol}.
  \item We see from F $\rightarrow$ f I v A w S x that w is in FOLLOW(A), x
  is in FOLLOW(S) and v is in FOLLOW(I)
  \item FOLLOW(P) = FOLLOW(A) since A $\rightarrow$ P. 
  \item FOLLOW(S') = FOLLOW(S) since S $\rightarrow$ S'.
  \item FOLLOW(I) includes FIRST(P) except $\epsilon$ since P $\rightarrow$ I P
  $| \epsilon$, and rule 3, wich makes FOLLOW(I) = i, v
\end{itemize}
\begin{tabular}{|c|c|c|c|}
\hline
NT & FIRST & FOLLOW \\
\hline
\hline
F  & f & \$ \\
A  & i, $\epsilon$ & w \\
P  & i, $\epsilon$ & w \\
S  & s & x \\
S' & s, $\epsilon$ & x \\
I  & i & i, v, w \\
\hline
\end{tabular}
\subsubsection*{LL(1) Parsing table}
Followed page 224, Alg. 4.31\\
IF A $\rightarrow$ b AND FIRST(b) contains c THEN add A $\rightarrow$ b to
M(A,c) and\\
IF FIRST(b) contains $\epsilon$, THEN add A $\rightarrow \epsilon$ to M(A,c)
M(A,b) M(A,c)\\
\begin{tabular}{|c|c|c|c|c|c|c|c|}
\hline
Non- & \multicolumn{7}{c|}{Input Symbol}\\
\cline{2-8}
Terminal & f & i & v & w & s & x & \$ \\
\hline
\hline
F  & F $\rightarrow$ f I v A w S x & & & & & & \\
\hline
A  & & A $\rightarrow$ P & & A $\rightarrow$ $\epsilon$ & & & \\
\hline
P  & & P $\rightarrow$ I P & & P $\rightarrow$ $\epsilon$ & & & \\
\hline
S  & & & & & S $\rightarrow$ sS' & & \\
\hline
S' & & & & & S' $\rightarrow$ sS' & S' $\rightarrow$ $\epsilon$ & \\
\hline
I  & & I $\rightarrow$ i & & & & & \\
\hline
\end{tabular}

\subsubsection{Parse tree for LL(1)}
\begin{tabular}{c}
\includegraphics[width=40px]{LL1-ParseTree/1.pdf}
\includegraphics[width=160px]{LL1-ParseTree/2.pdf}
\includegraphics[width=160px]{LL1-ParseTree/3.pdf}\\
\includegraphics[width=160px]{LL1-ParseTree/4.pdf}
\includegraphics[width=170px]{LL1-ParseTree/LL1-parse-tree.pdf}
\end{tabular}

\subsubsection{Buttom-up parsing}
\begin{tabular}{|l|r|l|}
\hline
Symbol & Input & Action\\
\hline
\hline
\$ &  f i v i i i i w s s s x \$& Shift f\\
\$ f & i v i i i i w s s s x \$& Shift i\\
\$ f i & v i i i i w s s s x \$& Reduce I $\rightarrow$ i\\
\$ f I & v i i i i w s s s x \$& Shift v\\
\$ f I v & i i i i w s s s x \$& Reduce P $\rightarrow \epsilon$\\
\$ f I v P & i i i i w s s s x \$& Shift i\\
\$ f I v P i & i i i w s s s x \$& Reduce I $\rightarrow$ i\\
\$ f I v P I & i i i w s s s x \$& Reduce P I $\rightarrow$ P\\
\$ f I v P & i i i w s s s x \$& Shift i\\
\$ f I v P i & i i w s s s x \$& Reduce I $\rightarrow$ i\\
\$ f I v P I & i i w s s s x \$& Reduce P I $\rightarrow$ P\\
\$ f I v P & i i w s s s x \$& Shift i\\
\$ f I v P i & i w s s s x \$& Reduce I $\rightarrow$ i\\
\$ f I v P I & i w s s s x \$& Reduce P I $\rightarrow$ P\\
\$ f I v P & i w s s s x \$& Shift i\\
\$ f I v P i & w s s s x \$& Reduce I $\rightarrow$ i\\
\$ f I v P I & w s s s x \$& Reduce P I $\rightarrow$ P\\
\$ f I v P & w s s s x \$& Reduce A $\rightarrow$ P\\
\$ f I v A & w s s s x \$& Reduce A $\rightarrow$ P\\
\$ f I v A & w s s s x \$& Shift w\\
\$ f I v A w & s s s x \$& Shift s\\
\$ f I v A w s & s s x \$& Reduce  S $\rightarrow$ s\\
\$ f I v A w S & s s x \$& Shift s\\
\$ f I v A w S s & s x \$& Reduce  S $\rightarrow$ S s\\
\$ f I v A w S & s x \$& Shift s\\
\$ f I v A w S s & x \$& Reduce  S $\rightarrow$ S s\\
\$ f I v A w S & x \$& Shift x\\
\$ f I v A w S x & \$& Reduce F $\rightarrow$ f I v A w S x \\
\$ F & \$ & Accept \\
\hline
\end{tabular}

\subsection{Symbol tables}
\subsubsection{Data structure}
Symbol tables are often represented as one or more hash tables. Hash tables
provide fast lookup, so that we have rapid access to the information about the
current symbol. We may use multiple hash tables to ease the parsing though
different scopes of the parsed program, eg. one hash table for each scope,
stored in a hierarchy.

\subsubsection{Function pointers}
When we allow function pointers in our language, we may hit several problems
during compile time. If the symbol table is just like a symbol table for
programing languages without function pointers, we cannot know if {\ttfamily
dostuff} is a variable or a function, neither do we know how many arguments it
take or if it returns anything. Since we cannot raise errors about things we
don't know about, we must just hope that the program runs correct at run-time.

\subsubsection{Extending the symbol table}
If we add type (FUNC / VAR / FUNC\_PTR), returns symbol (Y/N) and number of
arguments to the symbol table, then we may raise errors during compile time if the user has mixed
up types or number of arguments.

\subsubsection{Structure of a symbol table entry}
\begin{tabular}{|c|c|c|c|c|}
\hline
\textbf{Symbol} & \textbf{Mem loc} & \textbf{Type} & \textbf{Returns val} & \textbf{N
args}\\
\hline
f & dostuff & FUNC\_PTR & - & -\\
dostuff & {\ttfamily mem\_loc} & FUNC & false & 1\\
\hline
\end{tabular}

Where
\begin{itemize}
  \item Symbol is the referance symbol.
  \item Mem loc is the location in memory
where the function or the variable is located, or the referance symbol if it
references an other symbol.
\item Type is wherever this is a function, a variable or a function pointer.
\item Returns val is wherever the function, if it is a function, has a
return statement.
\item N args is the number of arguments the function takes, or 0 if it is not a
function.
\end{itemize}

\subsection{Syntax-Directed Translations}
\subsubsection{What is a SDD?}
Syntax-directed definition is our grammar with added support for attributes.
Attributes is any kind things we can store in memory, like numbers or strings.
This is usefull when we need to add semantic actions to our parse tree, eg.
create an AST, wich again is usefull for type checking and intermediate code
generation.

\subsubsection{L- and S-attributed SDD}
What is the difference between L-attributed and S-attributed syntax-
directed definitions? What does Bison support? Please explain.

L- and S-attributed SDD's are two classes of SDD's that do guarantee evaluation
order and does not permit dependency graphs with cycle.
S-attributed SDD's are 

\subsubsection{SDD and Grammars}
\begin{tabular}{lll}
E $\rightarrow$ & E + T & \{\}\\
E $\rightarrow$ & T & \{\}\\
T $\rightarrow$ & num , num & \{\}\\ 
T $\rightarrow$ & num & \{\}\\
\end{tabular}

\printindex
\end{document}
