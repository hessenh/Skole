%%This is a very basic article template.
%%There is just one section and two subsections.
\documentclass[english,a4paper]{scrartcl}

\usepackage{graphicx}
\usepackage{listings}
\usepackage{color}
\usepackage{makeidx}
\usepackage{hyperref}
\usepackage{parskip}
\usepackage{multirow}
\usepackage{tocloft}
\renewcommand{\cftsecaftersnumb}{\hspace{6em}}
\renewcommand{\cftsubsecaftersnumb}{\hspace{6em}}
\renewcommand{\cftsubsubsecaftersnumb}{\hspace{6em}}

\makeindex

\hypersetup{
    %bookmarks=true,         % show bookmarks bar?
    unicode=false,          % non-Latin characters in Acrobats bookmarks
    pdftoolbar=true,        % show Acrobats toolbar?
    pdfmenubar=true,        % show Acrobats menu?
    pdffitwindow=false,     % window fit to page when opened
    pdfstartview={FitH},    % fits the width of the page to the window
    pdftitle={TDT4205 Compilers - Exercise 6 - hvatum},    % title
    pdfauthor={Stian Hvatum},     % author
    pdfsubject={TDT4205 Compilers},   % subject of the document
    pdfcreator={Stian Hvatum},   % creator of the document
    pdfproducer={Stian Hvatum}, % producer of the document
    pdfnewwindow=true,      % links in new window
    colorlinks,       % false: boxed links; true: colored links
    linkcolor=black,          % color of internal links
    citecolor=green,        % color of links to bibliography
    filecolor=magenta,      % color of file links
    urlcolor=cyan           % color of external links
}

\definecolor{listinggray}{gray}{0.9}
\definecolor{lbcolor}{rgb}{0.9,0.9,0.9}
\lstset{
    keywordstyle=\bfseries\ttfamily\color[rgb]{0,0,1},
    identifierstyle=\ttfamily,
    commentstyle=\color[rgb]{0.133,0.545,0.133},
    stringstyle=\ttfamily\color[rgb]{0.627,0.126,0.941},
    showstringspaces=false,
    basicstyle=\tiny,
    numberstyle=\tiny,
    framexleftmargin=3pt,
    numbers=left,
    stepnumber=1,
    numbersep=15pt,
    tabsize=2,
    breaklines=true,
    prebreak = \raisebox{0ex}[0ex][0ex]{\ensuremath{\hookleftarrow}},
    breakatwhitespace=false,
    aboveskip={1.5\baselineskip},
    columns=fixed,
    upquote=true,
    extendedchars=true,
  	frame=l,
    sensitive=true,
}
\lstdefinelanguage{vsl}
  {morekeywords={FUNC,VAR,PRINT},
  sensitive=false,
  morecomment=[l]{//},
  morecomment=[s]{/*}{*/},
  morestring=[b]",
}


\renewcommand{\thesection}{PART \arabic{section}}
\renewcommand{\thesubsection}{Task \arabic{section}.\arabic{subsection}}
\renewcommand{\thesubsubsection}{\arabic{subsubsection}.}

\title{TDT4205 Compilers\\
\Huge Exercise 6}
\author{Stian Hvatum (hvatum)\\MTDT}

\begin{document}
\maketitle
\tableofcontents
\newpage
\section{Theory}
\subsection{Optimization}
\subsubsection{Easy and common}
One easy and common optimization done by compilers is common sub-expression
elimination. A common sub-expression is a an expression that occurs multiple
times either alone or within another expression during the space where the
involved variables are not redefined.

\begin{lstlisting}[language=C]
int main(int argc, char** argv)
{
    int a = 5;
    int b = 10;
    int c = 8;
    int d = a + b * c;
    int e = d - b * c;
}
\end{lstlisting}

At line 6 and 7 we see that \(b \cdot c\) occurs at both lines. We can reduce
the cost of mulitplying with this code:
\begin{lstlisting}[language=C]
int main(int argc, char** argv)
{
    int a = 5;
    int b = 10;
    int c = 8;
    int tmp = b * c;
    int d = a + tmp;
    int e = d - tmp;
}
\end{lstlisting}

We must notice that sometimes it is still cheaper to calculate the
sub-expression twice, since pre-calculating it may involve more memory useage
and eventually force the processor to use cache or main memory instead of
the faster registers.

\subsubsection{Hard to do}
Even for compilers utilizing Data-flow analysis, it is hard\footnote{Actually
the problem is NP-Complete and mathematically undecidable} to do a complete
general and correct calculation of Reaching Definitions.
Reaching Definitions is a way to find out which parts of a program that may be executed,
and which may not, according to value-dependent branching.

The reason why this is hard to do, is because it is (in general) impossible to
determin all possible values each variable may take given its surrounding
program. With data-flow analysis, we may determin some of these and maybe also
discover tautologies and contradiction, and thus we can eliminate some paths.
But since we cannot determin all possible values and paths, some paths will be
let though even though they never are executed.

\subsubsection{Optimalization impact}
\begin{description}
\item[Copy Progragation] \ \\
L7: g = f, where f = e from L6.
\end{description}
\begin{lstlisting}
a=1
b=2
c=3
d=a+x
e=b+c

g=e
g=d+y
a=b+c
\end{lstlisting}
\begin{description}
\item[Common subexpression elimination] \ \\
L5 and L9 both has b + c.
\end{description}
\begin{lstlisting}
a=1
b=2
c=3
d=a+x
t=b+c
e=t
f=e
g=f
g=d+y
a=t
\end{lstlisting}
\begin{description}
\item[Contant propagation] \ \\
a, b and c are all assigned to contants. There are used in L4, L5, L8, L9, and
can be replaced with the assigned number.    
\end{description}
\begin{lstlisting}
a=1
b=2
c=3
d=1+x
e=2+3
f=e
g=f
g=d+y
a=2+3
\end{lstlisting}
We could also used constant folding to eliminate all lines except L1, L4 and L8,
where we have unknown variables, given none of the other variables are used
after the given code block. If dead code elimination was used, we could
eliminated L7, since g is assigned to something else at L8.

\subsection{Misc}
\subsubsection{Array bounds checking}
To do bounds checking on array lookups and assignments, we need to know how
large the array is, and we need to check this lenght against the value of the
lookup-variable. This means that we need to store the array lenght somewhere,
most likely in the symbol table, and we need to add some extra assembly code
around each array lookup and assignment, unless the lookup is a constant or
given to be within bounds by external factors.

A high level abstraction of this:
\begin{lstlisting}[language=C]
    int array[5];
    
    /* No check is needed here, since index is constant */
    array[3] = 2;
    
   /* Here we need to check the index,
      and throw an error if out of bounds  */
   
    if (i >= 0 && i < 5)
        array[i] = 4;
    else
        throw index_out_of_bounds;
\end{lstlisting}

\subsubsection{Pointer arithmetics}
\paragraph{Live variables}
When we are looking for live variables, we usually go backwards and check where
the last occurence of each variable is used. Since pointers points to data
areas, there may be problems related to when that memory area was read. If you
write to a pointers data area, and read another pointers data area, they may
refer to the same area, and you cannot declare the first one as dead, even if
it is written only (and never ``used'').
\paragraph{Available expressions}
Regarding available expressions, we have sort of the same problem as with Live
variables. You may have great trouble knowing wherever your expression result is
stil valid if your data area has been modified by another pointer pointing to
the same location.

\subsubsection{Graph coloring and register spilling}
Register allocation is the problem of assigning registers to those results and
variables that is needed the most in the current state. Since you have a given
number of registers, and the expression calculated at the given state holds a
number of variables you can reduce this to the Graph coloring problem, where the
colors are registeres and the areas of the graph are the variables. Each
sub-expression has to be calculated without using the same register twice, hence
we cannot have two neighbouring areas that share the same color.

Register spilling is the phenomenon that occurs once we don't have enought
registers to calculate a problem. When register spilling occures, preformance
is greatly reduced. One of the main objectives of register allocation is to
minimize the occurences of memory spilling. 

\subsubsection{DAGs}
Compilers often represent basic blocks as DAGs because it eases some
optimizations, eg.
\begin{itemize}
  \item \textbf{Local common subexpression elimination}\\
  We will see from the DAG that if two calculations are the same, they will end
  up in the same node, and thus are equal at that point. 
  \item  \textbf{Dead code elimination}\\
  By removing all roots with no attached live variables until there are no more
  such nodes, we have effectivly removed all dead code from the block.
  \item  \textbf{Statement reoder}\\
  We can reorder statements that does not depend on each other, such that we get
  more common subexpressions that may be eliminated, eg. \\
  $a = b + c;$\\$d = b + f
  + c;$\\becomes\\ $a = b + c;$\\$d = b + c + f;$\\ where we see the common
  subexpression $b + c$. We must of course be avare of precedence and
  parentheses when doing such optimizations. The DAG will show us which
  statements that are dependent on each other.
  \item  \textbf{Simplify computation}\\
  We can do \emph{constant folding} and \emph{strenght reduction} on the
  expressions within the DAG. Eg. $x^2 = x \cdot x$, $2 \cdot x = x + x$ and
  $2 + 2 = 4$.
  The DAG is not strictly required for this, but it is very easy to make these
  changes on this kind of node-oriented data structures. Using this kind of
  sustitutions often render some of the used nodes as dead, so that they may be
  eliminated.
\end{itemize}
\end{document}
