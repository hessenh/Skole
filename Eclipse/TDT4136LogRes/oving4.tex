\documentclass[norsk,a4paper]{article}

% For norske bokstaver
\usepackage[T1]{fontenc}
\usepackage{babel}
\usepackage{wrapfig}
\usepackage{listings}
\usepackage{color}
\lstset{frame=shadowbox, rulesepcolor=\color{blue}}
\usepackage{qtree}
% \usepackage[a4paper,landscape]{geometry}
\usepackage[utf8]{inputenc}
\usepackage{parskip} % for litt mellomrom mellom avsnittene
\usepackage{amsmath,bussproofs}

\title{TDT4136 Logic And Reasoning Systems\\
{\Huge Exercise 4}}
\author{Stian Hvatum (hvatum)\\MTDT}

\begin{document}
\maketitle
\thispagestyle{empty}
\newpage
\pagenumbering{Roman}
\tableofcontents
\newpage
\pagenumbering{arabic}
\section{Task 1}

\framebox[\linewidth][c]{\parbox{0.9\linewidth} {
The following Prolog code defines a predicate \(p\). \\ 
 \(p(X,[X|Y]).\) \\
 \(p(X,[Y|Z]) :- p(X,Z).\)
}  }
\subsection{}
\framebox[\linewidth][c]{\parbox{0.9\linewidth} {
Show the proof trees of the queries\\
\(?- p(A,[1,2,3]).\)\\
\(?-p(2,[1,A,4]).\)
}}

\Tree [.\(p(A,[1,2,3])\)
    [.{\(p(A,[1,2,3])\)\\\(\Theta\)\{\(X/A,X/1,Y/[2,3]\)\}} \(A=1\) ]
        [.{\(p(A,[1,2,3])\)\\\(\Theta\)\{\(X/A,Y/1,Z/[2,3]\)\}}
                [.{\(p(A,[2,3])\)\\\(\Theta\)\{\(X/A,X/2,Y/[3]\)\}} \(A=2\) ]
                [.\(p(A,[2,3])\)\\\(\Theta\)\{\(X/A,Y/2,Z/[3]\)\}
                    [.{\(p(A,[3])\)\\\(\Theta\)\{\(X/A,X/3,Y/[]\)\}} \(A=3\) ]
                    [.\(p(A,[3])\)\\\(\Theta\)\{\(X/A,X/3,Y/[]\)\} 
                        [.\(p(A,[])\) False ]
                        [.\(p(A,[])\) False ]
                    ]
                 ]
        ]
    ]

\subsection{}
\(p\) representerer \(member(element,list)\), funksjonen sjekker om A er
element i lista.

\newpage
\section{Task 2}
\subsection{Kode}
Jeg har brukt \(insert\) fra forrige øving.
\lstinputlisting[language=prolog]{ex4t2.pl}
\subsection{Teori}
\subsubsection{Deloppgave 2d)}
Slow\_sort finner alle permutasjoner av lista, og sjekker om de er sorterte. Å
sjekke om en vilkårlig liste er sortert, tar \(O(N)\) tid. Å finne
permutasjoner alle \(N!\) permutasjonene tar \(O(N!)\). Vi må for hver
permutasjon sjekke om denne er sortert, til sammen gir dette en kompleksitet på
\(O(N\cdot N!)\).
\end{document}