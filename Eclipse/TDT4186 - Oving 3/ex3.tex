\documentclass[norsk,a4paper]{article}

% For norske bokstaver
\usepackage[T1]{fontenc}
\usepackage{babel}
\usepackage{wrapfig}
\usepackage{listings}
\usepackage{color}
\usepackage[utf8]{inputenc}
\usepackage{parskip} % for litt mellomrom mellom avsnittene
\usepackage{amsmath,bussproofs}

\definecolor{listinggray}{gray}{0.9}
\definecolor{lbcolor}{rgb}{0.9,0.9,0.9}
\lstset{
    backgroundcolor=\color{lbcolor},
    tabsize=2,
    rulecolor=,
    language=prolog,
        basicstyle=\scriptsize,
        upquote=true,
        aboveskip={1.0\baselineskip},
        columns=fixed,
        showstringspaces=false,
        extendedchars=true,
        breaklines=true,
        prebreak = \raisebox{0ex}[0ex][0ex]{\ensuremath{\hookleftarrow}},
        frame=single,
        showtabs=false,
        showspaces=false,
        showstringspaces=false,
        identifierstyle=\ttfamily,
        keywordstyle=\color[rgb]{0,0,1},
        commentstyle=\color[rgb]{0.133,0.545,0.133},
        stringstyle=\color[rgb]{0.627,0.126,0.941},
}

\title{TDT4186 Operating Systems\\
{\Huge Exercise 3}}
\author{Stian Hvatum (hvatum)\\MTDT}

\begin{document}
\maketitle
\thispagestyle{empty}
\newpage
\pagenumbering{Roman}
\tableofcontents
\newpage
\pagenumbering{arabic}
\section{Description of Code}
\subsection{CPU}

\subsection{IO}

\subsection{Simulator}
In the Simulator class, I added three variables,
\begin{enumerate}
     \item private CPU cpu 
     \item private long avgIoTime 
     \item private IO io 
\end{enumerate}

These variables represents the CPU, the I/O unit and the average I/O-time for
each operation. The I/O-time is given as a system parameter, but it was never
keept nor used (as I could find).

I initialized the CPU and I/O just like the memory-unit, but I also gave them a
handle for the eventQueue and the GUI, so that they could dispatch events and
display ``what's going on'' to the user.

Code:
\lstset{language=java}
\begin{lstlisting}
public Simulator(...) {
  ...
  memory = new Memory(memoryQueue, memorySize, statistics);
  cpu = new CPU(cpuQueue, maxCpuTime, statistics, eventQueue, gui);
  io = new IO(ioQueue, statistics, eventQueue, gui);
  this.avgIoTime = avgIoTime;
  ...
}
\end{lstlisting}

\subsection{Statistics}

\subsection{Event}
I added a toString-method to the Event-class, for easier use of debugger, stack
traces and outputs. This has nothing to do with the program logic, only for my
own understanding of the code.

\section{Experimenting with the system}
\subsection{System parameters and their impact on performance}
\subsection{Alternative algorithms}

\end{document}