\section{Creating the Decision Support System}
\subsection{Chosing a problem}
The first step in solving the task for creating a decision support system is
selecting a problem for the system to solve. I have chosen to let the system
help in deciding the every-day-decision of what to do now, for a
regular student. For this kind of system, it is really hard to get real values
for the utility function, and also difficult to set default probabilities for
the environment. These values and probabilites must also be adjusted for each
subject, as different persons have different personalities, preferences and
work-effectiveness.


I have chosen the following activities that the system will help you decide
between:
\begin{itemize}
  \item Watch a movie with someone
  \item Watch a movie alone
  \item Do homework
  \item Eat dinner with someone
  \item Eat dinner alone
  \item Go to sleep
  \item Go outside
  \item Do excercise
\end{itemize}
of coure, there are a lot more things that could be added, but I think these are
a representative overall of actions.

\subsection{Decision parameters}
Now that we have our problem defined, we need information about the current
state, in order to promote a good decision.

I have chosen some attributes that I think is important in order to decide what
to do, in the general context.

\begin{description}
\item[Lot to do] \hfill \\
Students have a lot to do, occasionally, but may also have a lot of spare time.
Because how much you have to do is important deciding if you should do homework,
I have included this parameter.

\item[Hungry] \hfill \\
Whenever you are hungry, it is a good idea to eat food.

\item[Sleepy] \hfill \\
You might be better off with sleeping than doing homework or doing excercises if
you are sleepy.

\item[Tired] \hfill \\
Tired is kind of a extended expression for sleepy, hungry, but also general
``don't want to do nothing''. I have included this as a way of telling you are
tired, without beeing explicitly sleepy. If you are sleepy, of cource, this
affects tired, unless specified otherwise.

\item[Social needs] \hfill \\
Everyone has social needs from time to time, but sometimes, we also prefer
beeing alone. This is important deciding wherever we are going to spend time
alone, or meeting other people.

\item[Weather] \hfill \\
The weather is more an environmental condition, but is also important for some
decisions. We might say that it is more pleasing going out on a sunny
afternoon, then if it is a full storm outside. I would also state that there are
somewhat better to do homework while it is raining outside, since it lowers the
desire to go outside.

\end{description}

\subsection{The effect of a choise}
In order to calculate the utility value of a choise, we have to look at its
consequences. I have chosen some attributes that work as outcome of a choise and
a state. The attributes are the following:

