\documentclass[english]{article}

% For norske bokstaver
% \usepackage[T1]{fontenc}
\usepackage{babel}
\usepackage[utf8]{inputenc}
\usepackage{parskip} % for litt mellomrom mellom avsnittene
\usepackage{amsmath}

\renewcommand{\thesection}{\Roman{section}}
\renewcommand{\thesubsection}{\alph{subsection}}

\title{TDT4171 Artificial Intelligence Methods\\
\Huge Exercise 1}
\author{Stian Hvatum (hvatum)\\MTDT}

\begin{document}
\maketitle
\line(1,0){340} % 120mm is aprox. width of text field
\section{5-card Poker Hands}
\subsection{Atomic events}
The number of different poker hands available, is the combination of 52 cards,
choosen 5 at a time. \footnote{If we take
the order into account, the number would be \(52 \cdot 51 \cdot 50 \cdot 49 \cdot
48 = 311875200\). }

\({{52}\choose{5}} = 2598960\)

\subsection{The probability of an atomic event}
The probability of each atomic event is equal, given the dealer is fear. This
means the probability of each event is
\(1/2598960 = 0.0000038477\)

\subsection{The probability of special hands}
\subsubsection{The probability of a Royal Straight Flush}
There are four possible different Royal Straight Flush-hands in poker. Since
their probability each are equal to all other possible hands, the probability of
one of them, are \(0.0000038477 \cdot 4 = 0.00015391\)

\subsubsection{The probability of a Three of a Kind}
In order to get Three of a kind, you need to get 3 cards of the same value, and
2 of any other value. First, we need to be given one card of value and color.
Then we need to be delt 2 more of this color, that is
\({{13}\choose{1}}{{4}\choose{3}} = 13 \cdot 4 = 52\). This is the number of
ways we can choose 3 cards of same value from a normal deck of cards. We need to
multiply this with the number of ways we can choose the rest of the other cards,
wich is to choose 2 cards with a different value, both with any color.
\({{12}\choose{2}}{{4}\choose{1}}{{4}\choose{1}} = 66 \cdot 4 \cdot 4 = 1056\)

The mathematical formula will then be:
	\({{13}\choose{1}}{{4}\choose{3}} \cdot {{12}\choose{2}}{{4}\choose{1}}^2 =
	52 * 1056 = 54912\)

\end{document}
