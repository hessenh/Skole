\documentclass[screen]{beamer}
\usepackage[T1]{fontenc}
\usepackage[latin1]{inputenc}
\usepackage{parskip}

% Bruk NTNU-temaet for beamer (her i bokmålvariant), alternativer er
% ntnunynorsk og ntnuenglish.
\usetheme{ntnuenglish}
 
% Angi tittelen, vi gir også en kortere variant som brukes nederst på
% hver slide:
\title[Effective groups]%
{Ground rules for effective groups}

% Denne kan du også bruke hvis det passer seg:
\subtitle{A compilation from "The Skilled Facillitator"}

% Angir foredragsholder, også en (valgfri) kortversjon i
% hakeparanteser først som kommer nederst på hver slide:
\author[S. Hvatum]{Stian Hvatum}

% Institusjon. Bruk gjerne disse slik det passer best med det du vil
% ha.  Valgfri kortversjon her også
\institute[NTNU]{Institutt for datateknikk og informatikk}

% Datoen blir også trykket på forsida. 
\date{20. februar 2013}
%\date{} % Bruk denne hvis du ikke vil ha noe dato på forsida.

% Fra her av begynner selve dokumentet
\begin{document}

% Siden NTNU-malen har en annen bakgrunn på forsida, må dette gjøres
% i en egen kommando, ikke på vanlig beamer-måte:
\ntnutitlepage

% Her begynner første slide/frame, (nummer to etter forsida). 
\begin{frame}
  \frametitle{Ground Rules for effective groups}

  \begin{enumerate}
    \item Test assumptions and inferences
    \item Share all relevant information
    \item Use specific examples and agree on what important words mean
    \item Explain your reasonining and intent
    \item Focus on interests, not positions
    \item Combine advocacy and inquiry
    \item Jointly design next steps and ways to test disagreements
    \item Discuss undiscussable issues
    \item Use a decision-making rule that generates the level of commitment needed
  \end{enumerate}

\end{frame}

\begin{frame}
    \frametitle{Why?}

    \begin{itemize}
        \item Diagnostics framework
        \item Gained group efficency
        \item Internal and external observations
        \item Must be a free and informed choise
    \end{itemize}

\end{frame}
\begin{frame}
    \frametitle{Test assumptions and inferences}

When two or more persons communicate, we are usually very quick to infere more meaning than communicated. The meaning that we assume may not be correct,
even though we ourselves are certain.

    \begin{itemize}
        \item Be aware of assumptions and inferences that you make during conversations
        \item Check your assumptions, ask if your perceptions are equal to the other persons communicated intent
        \item Allways state your inferences, such that missunderstandings does not occur
    \end{itemize}

\end{frame}


\begin{frame}
    \frametitle{Share all relevant information}

    \begin{itemize}
        \item If you have information that is relevant to the discussed topic, you must allways share it with the group.
        \item If your information is percieved by the others in the group as irrelevant, check with previous ground rule.
        \item If your information is in fact irrelevant, schedule the topic for later if important enough.
    \end{itemize}

\end{frame}

\begin{frame}
    \frametitle{Use specific examples and agree on what important words mean}

    \begin{itemize}
        \item If difficult words occur during conversation, it is important to explain the word such that everyone understands the meaning
        \item Use examples and counter-examples in a way that people both understands what a word means AND what it doesn't mean
        \item This empesizes the two previuous rules about sharing relevant information and assuming what people understands
    \end{itemize}

\end{frame}
\begin{frame}
    \frametitle{Explain your reasonining and intent}
    
    If one makes a statement about what the group should use their time on, without explaining both with what intention and why the action
    would help the case, the other group members will most likly assume and infere things that are wrong.

See how each of the previous rules
    will help to understand why this rule is imporant.

\end{frame}
\begin{frame}
    \frametitle{Focus on interests, not positions}

        If a group tries to decide when to have the next meating, and each of the members propose dates without telling
        why, it will be hard to satisfy all needs. One tends to focus on his own positions, that is, what is important for me.
\end{frame}
\begin{frame}
    \frametitle{Focus on interests, not positions cont...}


   \begin{block}{Example}
        Person A, B and C wants to meet for a discussion on buying a new car for the company. A wants to buy as fast as possible,
        B wants to do research before running to the store, and C wants to wait two more weeks, since then they have more funds to
        invest in a new car.

If A starts arguing on why they should start early, without telling with what interests he has, it will
        be very difficult for the two other persons to understand his statements. The best way to agree is to expose the interests in
        such a way that one can be informed on everyones interests and decide on the best common choise.
    \end{block}   

\end{frame}
\begin{frame}
    \frametitle{Combine advocacy and inquiry}

    To combine advocacy and inquiry means that one should explain ones thoughts and interests in such a way that everyone understands
    the details, and then encourage the group to give genuine inquirys, that is ask questions.

    \begin{description}
        \item[Genuine inquiry] is when the intent is pure learning, like "Can you explain more about that?"
        \item[Rethorical inquiry] is when the intent is to ease in your oppinion without making it direct.
        "Wouldn't it be better if we did it this way, now that the situation is like this?" is a generic example.
    \end{description}

\end{frame}

\begin{frame}
    \frametitle{Jointly design next steps ...}

Jointly design next steps and ways to test disagreements means to decide workflow before each task, seems like "omkransning".

Acctually, it's just that. Agree on why, what, how, such that everyone follows.

Also when a decision has been made, explicitly
ask everyone if they have comments and if they agree. That way, we don't assume quiet == agreement, but we test for disagreements.

\end{frame}

\begin{frame}
    \frametitle{Discuss the undiscussable}

\begin{itemize}
    \item Problems that makes yourself or others in the group "loose face" is likely to be avoided
    \item By asserting these issues, one will help the group more than by letting the issues grow
\end{itemize}

    By using the previous rules, one can point out the intent of asserting the problem. If the problem is let alone,
group splitting can occur, and that is very counter-effective. Usually, there is no one in the group that intends to
destroy the project, and if a person is ineffective, he or she needs to disclose the reason for that.

If a group member tries to fix these kind of issues outside the group, uncertenty is likely to arise within the
rest of the group, as they are not part of what is happening.

\end{frame}

\begin{frame}
    \frametitle{Use a decision-making rule that generates the level of commitment needed}

This basically is a pointer to the importance of how decision-making is done. There are presented four models:
\begin{description}
    \item[Consultative] Leader consults group, leader makes decision
    \item[Democratic] Voting, over 50\% wins
    \item[Consensus] All members totally agree
    \item[Delegative] A sub-group gets to decide
\end{description}

\end{frame}

\begin{frame}
    \frametitle{Use a decision-making rule cont...}

There is also mentioned five levels of commitment to a taken decision, but there are no direct mapping between commitment level and decision-making type.

\begin{description}
    \item[Internal commitment] Self driven, full commitment
    \item[Enrollment] Does all that is nesessary within the given role
    \item[Compliance] Does what is formally required
    \item[Noncompliance] Does work, but not good enough to follow formal requirements
    \item[Resistance] Undermines the work
\end{description}

\end{frame}
\begin{frame}
    \frametitle{Questions or comments?}

    I  do now assume total understanding of the presented material, unless I get questions and comments!

\end{frame}
    %  \begin{block}{Bokstittel}
    %    Comodo consequat.
    %  \end{block}
\end{document}
